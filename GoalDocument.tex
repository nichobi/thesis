\documentclass{article}

% Language setting
% Replace `english' with e.g. `spanish' to change the document language
\usepackage[UKenglish]{babel}

% Set page size and margins
% Replace `letterpaper' with `a4paper' for UK/EU standard size
\usepackage[a4paper,top=2cm,bottom=2cm,left=3cm,right=3cm,marginparwidth=1.75cm]{geometry}

% Useful packages
\usepackage{amsmath}
\usepackage{graphicx}
\usepackage[colorlinks=false, allcolors=blue]{hyperref}

\title{Goal Document: Translating Typing Rules into JastAdd}
\author{Nicholas Boyd Isacsson}

\begin{document}
\maketitle

\section{Involved parties}
\subsubsection*{Student}
  Nicholas Boyd Isacsson \\
  \href{mailto:nicholas@isacsson.se}{nicholas@isacsson.se}

\subsubsection*{Supervisor}
  Christoph Reichenbach \\
  \href{mailto:christoph.reichenbach@cs.lth.se}{christoph.reichenbach@cs.lth.se}

\subsubsection*{Examiner}
  Niklas Fors \\
  \href{mailto:niklas.fors@cs.lth.se}{niklas.fors@cs.lth.se}

\section{Background}
In programming languages there are many types of data, and for any real world use case there will be constraints upon these types, such as what types of input a function can contain.
Type checking is a method of verifying, either at runtime (dynamic checking) or compile time (static checking), that these constraints are held.

JastAdd is a meta-compilation system, allowing one to work with and extend abstract syntax trees with great flexibility. % and is extensively used within the SDE group at LTH.
However, implementing a type checker in JastAdd by hand currently requires a lot of boilerplate code, slowing down iteration.
Typing rules offer a condensed, standardised syntax for defining simple languages, allowing fast iterations with minimal work, but these rules then typically need to be implemented by hand.


\section{Project description}
% Build JastAdd Specification from Typing Rules
% Typing rules are an efficient way for expressing flow-insensitive program analyses.  The goal for this thesis project is to automatically translate typing rules into JastAdd rules, automatically selecting different type inference schemes (monomorphic, polymorphic, syntax-directed or not, ...).

The goal of this project is to translate the concisely formatted typing rules into a working type-checker in JastAdd.
We will start with a basic implementation able to handle simple syntax-directed rules.
Once this is working, there are two possible directions to take the project.
One is to support object-oriented languages, by adding support for subtyping and ad hoc polymorphism.
The other is to support a more functional approach by implementing a Hindley-Milner style type system.
Which of these routes to take will be decided based on the experience of implementing the basic system.

Research questions include what language features we can support and what approaches we can use to translate different cases and how best to leverage JastAdd's capabilities.

% deliniate scope, syntax-directed rules, need not be fully self-contained
% eventually break out into one of non syntax-directed rules Hindley-Milner
% or subtyping and adhoc polymorphism

% Research questions:
%   What can be translated
%   *How to translate*, different approaches


%\section{Motivation} % merge with background
% Why compile to JastAdd?


\section{Related work}
\begin{itemize}
  \item
    André Pacak, Sebastian Erdweg, and Tamás Szabó. 2020. A systematic approach to deriving incremental type checkers. Proc. ACM Program. Lang. 4, OOPSLA, Article 127 (November 2020), 28 pages. https://doi.org/10.1145/3428195 \\
    \url{https://szabta89.github.io/publications/inca-typechecking.pdf} \\
    A paper about implementing a compiler, capable of deriving incremental type checking from typing rules.
    Similar goal to ours, but with a significantly different compilation target, meaning we will likely differ in architecture and what trade-offs are made.

  \item
    Mikael Pettersson. 1996. A Compiler for Natural Semantics. In Proceedings of the 6th International Conference on Compiler Construction (CC '96). Springer-Verlag, Berlin, Heidelberg, 177–191. \\
    \url{https://link.springer.com/chapter/10.1007/3-540-61053-7\_61} \\

  \item
    Grigore Roșu, Traian Florin Șerbănută. An overview of the K semantic framework.
The Journal of Logic and Algebraic Programming, Volume 79, Issue 6, 2010, Pages 397-434, ISSN 1567-8326, https://doi.org/10.1016/j.jlap.2010.03.012. \\
    \url{https://www.sciencedirect.com/science/article/pii/S1567832610000160} \\

  \item
    Saioc, G., Hüttel, H. (2022). A Tool for Describing and Checking Natural Semantics Definitions of Programming Languages. FROM. \\
    \url{https://arxiv.org/abs/2209.09471}

\end{itemize}


\section{Evaluation}
The first step of evaluating the project is to see what language features it supports. % What are the challenges? What properties of typesystems make them difficult/easy to embed? Efficiency, spec size
Once the program is capable of compiling a basic language, we can also compare it to existing handwritten (or written ourselves) JastAdd specifications for those same languages.
Here we will evaluate the size of the finished specification as well as the performance of the type checking.

An additional point of comparison is Pacak et al.'s \emph{A systematic approach to deriving incremental type checkers}, to which we will compare supported features as well as output size and performance.
%  Case studies, implementation possible by hand or already finished implementations
%  Compare with handwritten implementations:
%    - Spec size
%    - Performance
%  Compare with datalog paper

\section{Methodology}
% Steps:
%   - Input language and parsing
%   - Simply typed lambda calculus
%   - Additional features
%   ...
%   - Evaluation
%   - Finalise the thesis document (collect bullet points throughout and organise them but don't write text)

The timeline for the project is spread out over 20 weeks, from now until the beginning of June.
The following steps are planned, with approximate timeframes:
\begin{itemize}
  \item Define an input language and handle parsing, 2 weeks
  \item Implement the basics required for simply typed lambda calculus, 2 weeks
  \item Implement as many additional language features as time allows, 8 weeks
  \item Evaluation, 2 weeks
  \item Finalise the thesis document, 6 weeks %(collect bullet points throughout and organise them but don't write text)
\end{itemize}


\end{document}
